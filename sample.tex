\documentclass{syntax}

\title{Paper title}
\date{February 15, 2019}
% Use your id for this
\author{12345678}

\begin{document}

\maketitle
% remove page style for first page
\thispagestyle{empty}

\begin{abstract}
  Hello.
\end{abstract}

\section{Introduction} \label{sec:intro}
Hi. This is a citation \citep{wallace}. This is another kind of citation
\cite{wallace}. Use the first, probably.

\section{Synonyms for ``Hello''}
Hello. Hi (see section \fref{sec:intro}). Hey. Yo.

\subsection{Debated synonyms} \label{sec:debated}
This is \fref{sec:debated}. It is also \Fref{sec:debated}. 

All\^{o}. A-hoy-hoy. Moshi moshi.

\begin{exe}
  \ex
    \begin{forest} 
      [Root
        [Left child]
        [Right child
          [Right child', roof]
        ]
      ]
    \end{forest} \label{tree:sample}
\end{exe}

\Fref{tree:sample} is an example.

\newpage

\section{Known antonyms}
Bye. Good bye. Bye-bye. See some more examples below of how to not say
``hello'':

\begin{exe}
  \ex[*]{Good morning.} \label{ex:not-antonym}
  \ex{Farewell.} \label{ex:antonym}
\end{exe}

Refer to \fref{ex:not-antonym} for something that isn't an antonym. 
And refer to \fref[vario]{tree:sample} for an unrelated reason.

\section{Sample IPA}
Stolen directly from \href{http://www.l.u-tokyo.ac.jp/~fkr/tipa/tipaman.pdf}{the manual for tipa},
the following are equivalent:

\begin{exe}
  \ex \textipa{[\textsecstress\textepsilon kspl\textschwa
  \textprimstress ne\textsci\textesh\textschwa n]}
  \ex \textipa{[""Ekspl@"neIS@n]}
\end{exe}

\bibliography{sample}

\end{document}
